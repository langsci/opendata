\documentclass[output=guidelines,guidelines] {langscibook}
\renewcommand{\lsSeries}{guidelines}
\title{Cookbook for Open Access books}
\author{Sebastian Nordhoff\newlineCover Language Science Press}
\begin{document}
\maketitle
\frontmatter 
\tableofcontents 
\mainmatter
\chapter{Introduction}
This book details the experiences and lessons learnt when setting up Language Science Press, in the period from 2012--2018. It makes recommendations for how to set up an open access book publisher. These recommendations are drawn from personal experience and have a practicioner's view. As a correlate, theoretical considerations and empirical evidence are backgrounded. This is on purpose. 

It is one of our goals that many XYZ Science Presses will spring up in the future, in other fields ranging from Archeology to Zoology. Obviously, every scientific field is different and has its own culture and its own communities of practice. It is hoped that this cookbook will help these initiatives in calibrating their setup based on our experiences. 

\section{Scope}
Language Science Press publishes open access books in linguistics. We do not have insights into closed access books or journal articles, although some of our findings can probably be transferred. 

Linguistics as a discipline is typically considered part of the humanities. This being said, it is often dubbed the ``humanities discpline closest to the natural sciences''. A number of practices in linguistics are closer to the natural sciences than to the humanities in general: linguists publish a lot of journal articles; peer review is generally required; linguistic articles can contain a lot of diagrams, charts, tables, and theorems; and the scientific community is organised on a world-wide scale, with English as the normal language of publication. It may or may not be the case that the leading role that linguistics has in open access among the humanities is somehow related to these practices. In general, authors in linguistics do not have to be convinced to publish open access; they demand it. This seems to be different in other fields of the humanities. This particular position of linguistics has to be taken into account when applying the recipes presented here. 


\chapter{Prime the pump}

The most important take-home message is: you have about 2-3 years after the launch to build up enough prestige to keep rolling. 
After that period, you either have a steady stream of submissions, or your project is all but dead. 

Success begets success, and nothing attracts the crowd like the crowd. You want to be seen as the place to be, as the place where everyone flocks to. Every new publication should lead to a number of follow-up proposals from readers who have become interested in the press. The problem with books is that they take a year to write and a year to review and produce. This means that the follow-up books will come out about 2 years after the book which prompted them. In other words, the period from generation to generation is two years. 

For Language Science Press, we can distinguish three ``waves'' of proposals: 

\begin{enumerate}
 \item starting books (5). These are manuscripts invited by the press directors and are available for publication within the first year after launch
 \item second wave (12). These are submissions from authors who have seen the publication of the first books.
 \item third wave and beyond (50): Submissions from authors who have seen a second-wave or later book. 
\end{enumerate}
\todo{check numbers}

Your operations can be seen as steady when you have reached the third generation of proposals, i.e. follow-up submissions on follow-up books on the starting books. At that point in time, the generations become less clearly distinguished, and quick books from the fourth generation might come out earlier than latecomers from the third generation. Your initial funding should allow you to get to the publication of a third generation book.

A number of conclusions follow from these premises:

\begin{enumerate}
 \item You should have a book publication out ONE DAY AFTER your initial funding starts in order to attract second-generation proposals early. 
 \item Your first publications should be of very high quality, and should be as speedy as possible without sacrificing quality. A delay in first-generation books will set you back for all future generations, in terms of time and in terms of quantity.
 \item You must make sure that the starting books get maximal exposure in all relevant venues in order to produce follow-up submissions. For the first three generations, you want exponential growth. After that, linear growth will do. 
\end{enumerate}

Normally, you will not be able to adjust your staffing according to the number of books. This means that in the beginning, there will be too much staff, i.e. relatively fewer books per person to work on than later on. Use this leeway to create beautiful hand-crafted books and as pleasant as an experience for the authors as possible. Once the ball gets rolling and you attract more submissions, you will have to scale down that level of dedication, but at that point in time, you will hopefully be in a position to be selective with regard to the submissions you accept. 

In the beginning, we had some edited volumes where chapters were written in various versions of Microsoft Word, with no particular template, and no consistent or coherent formatting. We adapted all of those to a uniform style and spent enormous amounts of time on this. As a result, we attracted more submissions, and we are now able to enforce our templates and guidelines, and can afford refusing submissions which do not comply. As the number of books in production goes up, the time we have available for each book goes down, but at the same time, the submissions are in much better shape and require less time overall as well. 

\chapter{Tasks}
This chapter will detail a number of tasks which a fledgling project has to deal with. For each task, I will give some background and give some recommendations. The tasks to be discussed are:

\begin{enumerate}
 \item Creating prestige 
 \item Acquisition of manuscripts
 \item Discoverability
 \item Book production
 \item Software
 \item Accountability
 \item Financial matters
 \item Legal matters
 \item Governance structure
 \item Community building
 \item Branding
 \item Building a network
 \item Competition
\end{enumerate}


\section{Creating prestige}
This task is the most important one. It is the single item which will determine success or failure of your enterprise. 
Prestige is a virtuous circle: with better prestige, you get better submissions, which lead to better prestige etc. The reverse is also true. You must invest as much as you can in starting as high up on the prestige ladder as you can possibly afford. Try to take over the publishing landscape working from the top of the quality pyramid to the bottom, rather than the other way round.

Obviously, in the beginning, you have zero books. This means that you have nothing to show, and you must signal prestige in another way. We can name six ways: %TODO check whether this corresponds to sectioning below
bold claims, 
selectivity, 
big names, 
large crowd, 
CI, 
innovation. 

\subsection{Bold claims}
Make your claim to prestige very clear, very loud, and very often. The important thing is that you are aware that this is a \textsc{claim}, not a truth. In the case of Language Science Press, we asserted that we want to play in the Champions League of publishers and that we want to be better than de Gruyter (a leading publisher in linguistics). Have a couple of talking points where you want to be better in case someone asks you. For instance, faster turnaround time; better typography; better integration of multimedia. It is OK if people think the claim is preposterous, actually, this shows that your claim is just right. 

\subsection{Selectivity}
Assert that you are selective and that not everybody will be able to publish with you. Highlight your rigorous quality assurance principles. Disclose your rejection rates. The point is not so much in rejecting bad manuscripts, but in deterring authors from submitting low quality manuscripts in the first place. The clearer you make that low quality manuscripts will be rejected, the less you will get of them. And low quality manuscripts cause more work than better ones, for less reward (or none at all if rejected). 
At all costs you must avoid as being seen a publisher of last resort. The first books will set the standards for the future submissions, so better be strict. When you get a good submission, do everything to publish it quickly to attract follow-ups, but do not try to generate follow-ups on bad submissions: they will be bad as well. 


\subsection{Big names}
Obviously, you have to back up your claim. Remember, you still have zero books. Researchers associate prestige with leaders in the field. Establish contact with all leading scholars in your field and try to get them to publicly support your enterprise. Go for people the general public might recognize. Create a list of supporters and list those researchers at the very top.  

Get people from abroad. Prestige grows with the square of the distance. Endorsements from another city are great, from another country even better, but you should really try to get endorsements from researchers from several continents. 

For Language Science Press, we got support from Luc Steels and Steven Pinker, who have both been featured in the popular press. Their support was crucial in substantiating our claims to prestige. 

\subsection{Number of supporters}
Next to the big names (quality), you can also signal prestige by sheer quantity. Nothing attracts the crowd like the crowd, so open up your supporters list for all interested people and show the large backing you have from the community. Interested authors will see how large a supporter base you have and conclude that those people cannot possibly all be wrong. This is again a virtuous cycle: if you have many supporters, people will be more likely to join the crowd, increasing your number of supporters even more etc. Again, this means that you should make sure you have a sizable number of supporters to start. 


 
\subsection{CI}
CI is a technical term from design and marketing and stands for `Corporate Identity'. This term is used even if the entity under discussion is not a corporation. Your project should have a professional and concerted appearance, online and offline. Hire a professional designer to assist you. Pay for a professional logo. Be strict. Do not allow deviations from the CI, especially in the beginning.

In the case of Language Science Press, the designer advised on the choice of fonts, colours and book layout. All books use the same layout and the same title page. The colours used for the books series are also used on the website and in flyers. The font used is the same for books and flyers etc. This makes it easy to recognise some item as coming from Language Science Press. The uniform appearance across formats suggests stability and seriousness, even if no actual book has been published yet. 

\subsection{Innovation}
Open Peer Review%TODO

\subsection{Content}
The seventh component to signal prestige is the actual content of your first book. The six items discussed before are only promises. With the first book, you have to deliver on them. 

There is trade-off between quality and speed of publication. In the beginning, you have less books, hence more time per book. Make sure your first book is picture perfect, and add some bells, whistles and gongs. These do not really add to the scientific quality of the book, but they underscore that you take your new publishing business seriously. At Language Science Press, we have clickable cross-refences. For the string "as seen in Table 2", you can click on the "2" and your pdf viewer will jump to Table 2. Normally, Table 2 is in the direct vicinity of the text passage, hence there is little need for the hyperlink, but this is some extra feature which we highlighted. Another item is that most of our geographical maps are searchable. This means if you search "Somalia" in the pdf, you will find the string "Somalia" in the prose text, but also in maps of Africa where it occurs. Finally, all of our books have three indexes (Subject index, name index, language index). Most people probably never use the index, but rather hit CTRL+F when they want to find a particular term in the pdf, but the creation of the index was a proxy for us to signal prestige. In order to make claims to prestige, you must make absolutely clear that you do not simply provide printouts of random Microsoft Word documents as they are submitted. 


 

 
\section{acquisition}
\subsection{getting authors}
\subsubsection{early researchers}
                    OA citation advantage
                    idealism
\subsubsection{senior researchers}
                    leadership
\subsubsection{community involvement (see below)}
\section{discoverability}
\subsection{getting readers}
\subsubsection{digital}
                    web site
                        staple
                        responsive
                    blog
                        engaging
                        not redundant to web site
                    platforms
                        GoogleBooks
                        Zenodo
                        Amazon
                        Wikipedia
                        DOAB
                        OAPEN
                        Library catalogs
                    social media
                        Twitter
                            library folks
                        Facebook
                            researchers
\subsubsection{analog}
                    book stands
                    cofnerence ambassadors
\subsubsection{book reviews}
                    per request
                    no tracking
\subsubsection{readers become authors}
\subsection{identifiers}
\subsubsection{DOIs}
                    book
                    chapter
\subsubsection{ISBNs}
\subsubsection{keywords }
                    unclear if necessary for OA
\section{technology}
\subsection{requirements}
\subsubsection{catalog}
                    world
\subsubsection{workflow}
                    submission
                    review
                    publication
\subsubsection{access control}
                    closed access
                    draft versions
\subsection{book production}
\subsubsection{workflow}
                    ingestion 
                        specified format
                        random format
                    generation
                        latex
                        makefiles
                        pgfplots
                        tikz
                    outlet
                        digital
                        pod
\subsubsection{formats}
                    latex
                    docx/odt
                    indesign
                    markdown
                    pdf
                    xml
\subsubsection{elements}
                    frontmatter
                        title
                        frontispiece
                        series information
                        Innentitel
                    backmatter
                        indexes
                        references
                        back title
                    mainmatter
                        text
                        heading
                        crossrefs
                        tables
                        figures
                        maps
                        charts
                        discipline specific elements
                            linguistic examples
                            proofs
                            theorems
                    semantic markup
                        separation of meaning and presentation
\subsubsection{automation}
                    standardization
                        templates
                        skeleton
                        guidelines
                    generation
                        toc
                        references
                        crossrefs
                        indexes
                    distribution
                        website upload
                        pod upload
                        zenodo upload
                    metadata
                    makefiles
\subsubsection{distribution}
                    digital
                        Web site
                        GoogleBooks
                        PaperHive
                    analog
                        pod
                            international
                            quality
                            speed
                            ease
                        Amazon
                        Book stores
\subsubsection{archiving}
                    DOAB
                    OAPEN
                    Zenodo
\subsubsection{services}
                    copy-editing
                    typesetting
                    indexing
                    service providers
                        ubiquity
                        India
                        free lancers
\section{software}
\subsection{in-house}
\subsubsection{CMS}
                    OMP
                    GitLab
                    RUA
                    janeway
                    jekyll
\subsubsection{GUI}
                    Kile
                    Texstudio
                    Inkscape
                    InDesign
                    Word
                    OpenOffice
\subsection{webservices}
\subsubsection{GitHub}
\subsubsection{Overleaf}
\subsubsection{Paperhive}
\subsubsection{Zenodo}
\subsubsection{gitbook}
\section{accountability}
\subsection{monitoring}
\subsubsection{internal}
\subsection{reporting}
\subsubsection{external}
\subsection{financial}
\subsubsection{see business model and spread sheet}
\subsection{social}
\subsubsection{transparency}
                    statistics
                        downloads
                        sales
                        submissions
                        acceptance/rejection
                        publications
                        duration
                        gender balance
                        geographical balance
\subsection{technological}
\subsubsection{GitHub}
\subsubsection{blog}
\subsection{organizational}
\subsubsection{work flows}
\subsubsection{guidelines}
\subsubsection{screencasts}
\subsubsection{upate interval}
\section{money}
\subsection{Revenue}
\subsubsection{refer business model}
\subsubsection{major}
                    BPCs
                    Print margin
                    IMs
                    donations
                    individual memberships
\subsubsection{minor}
                    affiliates
                    prizes
                    flattr
                    vg wort
                    grant money
\subsection{Expenses}
\subsubsection{refer business model and spreadsheet}
\subsection{Liquidity}
\subsubsection{loans}
\subsubsection{print margins}
                    quarter
                    month
\subsubsection{IMs}
                    year
\subsubsection{BPCs}
                    unclear
\subsubsection{project money}
                    unclear
\subsection{taxation}
                0
                7
                19
\subsection{Konten}
\subsubsection{Unikonto}
\subsubsection{Geschäftskonten}
\subsubsection{Kreditkarte}
\section{legal}
\subsection{legal forms}
\subsubsection{Uni}
\subsubsection{Ltd}
\subsubsection{Association}
\subsubsection{get advice}
\subsection{tax}
\subsection{brands}
\subsubsection{register brand}
                    specific name
                    geographical scope
                    get advice
\subsubsection{protect brand}
\subsubsection{FAIR OA principles}
\subsection{IPR}
\subsubsection{author contracts}
\subsubsection{FAIR principles}
\subsection{pod}
\subsubsection{non-exclusive}
\subsection{hosting}
\subsubsection{legal requirements of your jurisdiction}
                    betreiber
                    Ansprechpartner
                    terms and conditions
                    privacy
                        DSGVO
                        analytics
                        permission
\section{Governance}
\subsection{internal}
\subsubsection{advisory board}
\subsubsection{press directors}
\subsubsection{series editors}
\subsubsection{coordinator}
\subsection{external}
\subsubsection{service providers}
\subsubsection{customers}
\subsection{Mehrfachrollen}
\subsection{series model}
\subsection{anbindung}
\subsubsection{faculty}
\subsubsection{library}
\subsubsection{IT services}
\subsubsection{BgA}
\subsection{coordinator}
\subsubsection{liaison}
\subsubsection{scheduling}
\subsubsection{prellbock}
\subsubsection{advocacy}
\subsubsection{contact}
\subsubsection{inquiry}
\subsection{organigram}
\subsection{orga}
\subsubsection{passwords}
\subsubsection{wiki}
                    howto
\subsubsection{guidelines}
\subsubsection{update interval}
\section{community involvement}
\subsection{different seniority levels}
\subsubsection{junior}
\subsubsection{senior}
\subsection{community proofreading}
\subsection{community typesetting}
\subsection{community illustrations}
\subsection{series editors meeting}
\subsection{social media}
\subsection{inquiries}
\subsection{newsletters}
\subsubsection{sparse}
\subsection{gamification}
\subsection{Tshirts}
\subsection{mugs}

stability 
\section{Markenbildung}
\subsection{Claims}
\subsubsection{Champions League}
\subsection{Design}
\subsubsection{CI}
\subsection{wurfmaterial}
\subsubsection{none}
\subsection{social media}
\subsection{conferences}
\subsubsection{discipline}
                    acquisition
                    hard copies
\subsubsection{OA activism}
                    funding
\section{network}
\subsection{lobbying}
\subsubsection{government}
                    city 
                    state
                    national
                    EU
                        OpenAire
\subsubsection{Zivilgesellschaft}
                    okf
                    wikimedia
                    CCC
                    ACLU
                    EFF
\subsection{other disciplines}
\subsubsection{thefife}
\subsubsection{archeology}
\subsubsection{jura}
\subsubsection{romanistik}
\subsubsection{computer science}
\subsection{teaching}
\subsubsection{grad schools}
\subsubsection{MBA, IT, Library Science}
\section{competition}
\subsection{be nice}
\chapter{readings}
\section{reports}
\subsection{EU}
\subsection{ithaca}
\subsection{Knowledge Exchange}
\subsection{Knowledge Unlatched}
\chapter{timelines}
\section{road maps}
\chapter{Zeitfresser}
\section{references}
\section{edited volumes}

failure is impossible
series concept
\end{document}


